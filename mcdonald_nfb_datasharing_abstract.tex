This paper describes cross-sectional functional magnetic resonance imaging (fMRI) data from two block-design tasks, a resting state fMRI scan, and a default mode network (DMN) neurofeedback paradigm, along with accompanying behavioral and cognitive measures in an ongoing study. We report technical validation from n=125 participants of the final targeted sample of 180 participants. Each session includes acquisition of one whole-brain anatomical scan and whole-brain echo-planar imaging (EPI) scans, acquired during the aforementioned tasks and resting state. The data includes several self-report measures related to perseverative thinking, emotion regulation, and imaginative processes, along with a behavioral measure of rapid visual information processing. The present data are linked to the enhanced Nathan Kline Institute, Rockland Sample and builds on the comprehensive neuroimaging and deep phenotyping available therein. As limited information is presently available about individual differences in the capacity to directly modulate the default mode network, these data provide a unique opportunity to examine DMN modulation ability in relation to numerous phenotypic characteristics. 