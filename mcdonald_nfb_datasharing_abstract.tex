This paper describes a repository of openly shared data from an experiment to assess inter-individual differences in default mode network (DMN) activity. This repository includes cross-sectional functional magnetic resonance imaging (fMRI) data from the Multi Source Interference Task, to assess DMN deactivation, the Moral Dilemma Task, to assess DMN activation, a resting state fMRI scan, and a DMN neurofeedback paradigm, to assess DMN modulation, along with accompanying behavioral and cognitive measures. We report technical validation from n=125 participants of the final targeted sample of 180 participants. Each session includes acquisition of one whole-brain anatomical scan and whole-brain echo-planar imaging (EPI) scans, acquired during the aforementioned tasks and resting state. The data includes several self-report measures related to perseverative thinking, emotion regulation, and imaginative processes, along with a behavioral measure of rapid visual information processing. 

Technical validation of the data confirms that the tasks deactivate and activate the DMN as expected. Group level analysis of the neurofeedback data indicates that the participants are able to modulate their DMN with considerable inter-subject variability. Preliminary analysis of behavioral responses and specifically self-reported sleep indicate that as many as 73 participants may need to be excluded from an analysis depending on the hypothesis being tested. 

The present data are linked to the enhanced Nathan Kline Institute, Rockland Sample and builds on the comprehensive neuroimaging and deep phenotyping available therein. As limited information is presently available about individual differences in the capacity to directly modulate the default mode network, these data provide a unique opportunity to examine DMN modulation ability in relation to numerous phenotypic characteristics. 