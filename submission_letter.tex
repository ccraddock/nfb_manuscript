\documentclass{article}

\usepackage{lipsum}
% \usepackage{layout}

\newcommand{\ts}{\textsuperscript}
%\usepackage{fontspec}
%\setmainfont{Calibri}
\providecommand\CMItoname{}
\providecommand\CMItoaddress{}
\providecommand\CMItoopeningname{Editor}
\providecommand\CMIreline{Data note submission to Neuro{I}mage}
% Define CMI official colors
\usepackage[dvinames]{xcolor}
\definecolor{CMIcrimson}{RGB}{152,30,50}
\definecolor{CMIgray}{HTML}{5e6a71}
\definecolor{NYpurple}{HTML}{402467}

% Measurements are taken directly from the guide
\usepackage[top=2in,left=1in,bottom=1.5in,right=1in, foot=12pt]{geometry}
\usepackage{graphicx}
\usepackage[colorlinks=false,
            pdfborder={0 0 0},
            ]{hyperref}
\usepackage[absolute]{textpos}
\usepackage{ifthen}
\usepackage{soul}


% --- For placement of the horizontal line
\usepackage{tikz}
\usetikzlibrary{calc}

% --- A nice serif font (palatino), but not the prescribed nonfree ITC stone
%\usepackage[sc,osf]{mathpazo}
\linespread{1.05}

% Remove paragraph indentation
\parindent0pt
\setlength{\parskip}{0.8\baselineskip}
\raggedright
\pagestyle{empty}
% Ensure consistency in the footer
\urlstyle{sf}

\providecommand\CMIfromname{R. Cameron Craddock, PhD}
\providecommand\CMIfromtitle{Director, Computational Neuroimaging Lab}
\providecommand\CMIfromdegree{Ph.D.}
\providecommand\CMIfromdept{Center for Biomedical Imaging and Neuromodulation}
\providecommand\CMIfromaddress{140 Old Orangeburg Road, Orangeburg, New York, 10962}
\providecommand\CMIfromtel{(845) 398-5822}
\providecommand\CMIfromfax{}
\providecommand\CMIfromemail{\url{cameron.craddock@nki.rfmh.org}}
\providecommand\CMIfromweb{\url{computational-neuroimaging-lab.org}}
%\providecommand\CMIfromweb{\url{childmind.org}}
%\providecommand\CMItoname{Dr.~Wiseman}
%\providecommand\CMItoaddress{Department of Wisemen\\
%                            Wise State University \\
%                            Wisetown, Wise\ \ 12345-6789\\
%                            USA}
\providecommand\CMIdate{\today}
\providecommand\CMIopening{Dear \CMItoopeningname,}
\providecommand\CMIclosing{Sincerely}
% Update this and the next line to the correct path
%\providecommand\CMIsignaturefile{AleeSignatureVector}
\providecommand\CMIsignaturefile{cc_signature}
\providecommand\CMIlogofile{NKI_header}
\providecommand\CMIenclosure{}

\usepackage{fancyhdr}
\thispagestyle{fancy}

\renewcommand{\footrulewidth}{0pt}
\fancyfoot{}
\fancyfoot[C]{%
   {
   	   \begin{center}\textbf{\color{NYpurple}\sffamily A RESEARCH INSTITUTE OF THE OFFICE OF MENTAL HEALTH\\\rule{\textwidth}{1pt}}
   \footnotesize\color{CMIgray}\sffamily
   140 Old Orangeburg Road, Orangeburg, NY 10962 $\vert$ (845) 398-5500 $\vert$ Fax: (845) 398-5510 $\vert$ omh.ny.gov
   \end{center}
   }%\color{black}
   % \begin{textblock*}{7.5in}[0,0](.5in,10.25in)   % 6.375=8.5 - 1.5 - 0.625
   %     \sffamily
   %     % \hfill \color{CMIgray} \CMIfromdept
   %     % \\ \hfill \CMIfromname, \CMIfromdegree
   % 	   \begin{centering}
   % 		   \textbf{A RESEARCH INSTITUTE OF THE OFFICE OF MENTAL HEALTH}
   % 	   \end{centering}
   % \end{textblock*}
}

\fancyhead{}
\fancyhead[L]{%
    \begin{textblock*}{7.5in}[0,0](.5in,.5in)
        \includegraphics[width=7.5in]{\CMIlogofile}
    \end{textblock*}
	}
\renewcommand{\headrulewidth}{0pt}


\AtBeginDocument{
    % Text lines should be less than 6in long
    % \newgeometry{top=2in,left=1in,bottom=1in,right=1in}

    \CMIdate
    \bigskip

    \CMItoname\ifthenelse{\equal{\CMItoname}{}}{}{\\}
    \CMItoaddress\par
    \bigskip
	\ifthenelse{\equal{\CMIreline}{}}{}{\textbf{RE: \CMIreline}\par\bigskip}

    \ifthenelse{\equal{\CMItoopeningname}{}}{To whom it may concern:\par}{\CMIopening\par}
	
    }

\AtEndDocument{
    \par\vspace{2ex}
    \CMIclosing,

    \ifthenelse{\equal{\CMIsignaturefile}{}}{\bigskip\bigskip}{\includegraphics[width=3.5in]{\CMIsignaturefile}\\}%[-0.2\baselineskip]}

    \CMIfromname \\
    \CMIfromtitle\ifthenelse{\equal{\CMIfromtitle}{}}{}{\\}
	\ifthenelse{\equal{\CMIfromemail}{}}{}{Email: \CMIfromemail\\}
	\ifthenelse{\equal{\CMIfromemail}{}}{}{Phone: \CMIfromtel\\}
	\ifthenelse{\equal{\CMIfromfax}{}}{}{Fax: \CMIfromfax\\}

    \CMIenclosure
}

\begin{document}
\sffamily

I am pleased to submit the manuscript titled ``Real-time fMRI Neurofeedback Based Stratification of Default Network Regulation Neuroimaging Data Repository'' for consideration for publication in Neuro{I}mage. This manuscript describes a data-sharing repository that consists of multimodal neuroimaging data from an experiment designed to evaluate default mode network (DMN) regulation. We provide a detailed description of how the data is collected and a preliminary technical validation of the various tasks and behavioral assessments.

The default mode network (DMN) has been implicated in a variety of mental health disorders, including ADHD, anxiety, depression, and posttraumatic stress disorder. While early studies have shown differences in DMN connectivity in these disorders, converging research suggests that the activity of the network is more relevant. Specifically, the DMN does not turn off during task performance as expected, leading to a hypothesis of DMN dysregulation. At this time, there is no clear understanding of the mechanism underlying DMN dysregulation or the symptoms that it confers. Instead researchers tend to explain the phenomena based on the disorder it is being associated with. For example, in depression DMN dysregulation is believed to reflect rumination, in ADHD it reflects inattention, and in other disorders it reflects perseveration.

To better understand DMN regulation and its role in phenotypic variation, my colleagues and I have designed a real-time fMRI based neurofeedback task. We have combined this paradigm with fMRI tasks that activate and deactivate the DMN, to form a multidimensional characterization of DMN function. Additionally, we have collected a vast range of assessments to measure phenotypes that are commonly associated with DMN regulation. We are currently sharing this data, without restriction, as it is being collected. The repository currently contains data from 125 participants, 21 – 45 years old, with a variety of clinical and subclinical psychiatric symptoms. Data collection for this experiment is ongoing and will hopefully reach 180 participants in total.

Although Neuro{I}mage does not have a formal data descriptor article type, it has published many similar articles in the two recent special issues on data sharing repositories. We have discussed this manuscript’s suitability for Neuro{I}mage with Dr. Peter Bandettini, who thinks that it would be a good fit. We believe that the novelty of the experiments used to collect the data, and the overall value of the data to the broad neuroimaging community, make this manuscript a great candidate for publication in Neuro{I}mage.

This manuscript is an original work. It has been submitted for preprint in bioRxiv, but has not been published in whole or in part in any other journal. The authors have no conflicts of interest to disclose. 

We propose the following reviewers, who do not have a conflict of interest with the authors, to assess the quality of the described experiment

\begin{itemize}
\item Jessica Andrews-Hanna, PhD, Jessica.Andrews-Hanna@Colorado.edu
\item Guiseppe Pagnoni, PhD, gpagnoni@gmail.com
\item Frank Scharnowski, PhD, frank.scharnowski@uzh.ch
\item R. Nathan Spreng, PhD, rns74@cornell.edu
\end{itemize}

\end{document}
